\Chapter{Koncepció}

\Section{A fejezet célja}

Ebben a fejezetben a dolgozat témáját körülhatároljuk, bemutatjuk a releváns projektmenedzsment rendszereket, azok főbb funkcióit, előnyeit és hátrányait, valamint azt, hogyan kapcsolódnak az Acxor fejlesztéséhez. A cél nem a saját eredmények részletezése, hanem egy átfogó, irodalmi és technológiai alap bemutatása, amelyre a dolgozat további részei építenek. A hivatkozások jelentős része ebben a fejezetben kerül alkalmazásra, például \cite{coombs1987markup}.

\Section{Tartalom és felépítés}

A fejezet tartalma négy fő területre koncentrál: irodalomkutatás, technológia, piackutatás és követelmény specifikáció. Az irodalomkutatás során áttekintjük a legfontosabb projektmenedzsment rendszereket, a technológiai részben bemutatjuk a rendszerek főbb funkcióit, a piackutatás során összehasonlítjuk a meglévő megoldásokat, míg a követelmény specifikációban az Acxor fejlesztés céljait és alapelveit foglaljuk össze.

\Section{Projektmenedzsment rendszerek áttekintése}

\SubSection{JIRA}

A JIRA egy komplex, vállalati szintű projektmenedzsment és issue tracking rendszer. Kiemelt jellemzői:

\begin{itemize}
\item Issue tracking, Agile metodikák (Scrum / Kanban) támogatása
\item Workflow testreszabás
\item Jogosultságkezelés
\item Riportok és dashboardok
\item Széleskörű integrációk (Confluence, Bitbucket, GitHub, Slack, Teams)
\end{itemize}

A JIRA rugalmas és szinte bármi testreszabható, ami előnyt jelent nagyvállalati környezetben, ugyanakkor komplexitása miatt tanulási görbéje meredek. A rendszer hierarchiája Epic–Story–Task–Subtask struktúrát követ, ami előnyös a nagy csapatok számára. Automatizmusai trigger–condition–action logikán alapulnak, amely lehetővé teszi például automatikus állapotváltást, assignee hozzárendelést, vagy e-mail értesítések küldését.

Az Acxor fejlesztésében a JIRA inspirációként szolgált a funkcionalitás és a workflow kezelés terén, de egyszerűsített felhasználói felületet és átláthatóbb jogosultsági modellt alkalmaz.

\SubSection{Trello}

A Trello egy intuitív, kártya-alapú projektmenedzsment rendszer, amely kisebb csapatoknak vagy egyszerűbb folyamatokhoz ideális. Főbb jellemzői:

\begin{itemize}
\item Board → List → Card → Checklist vizuális hierarchia
\item Drag \& drop alapú kezelés
\item Minimális konfiguráció és real-time szinkron
\item Egyszerű automatizmusok (Butler)
\end{itemize}

A Trello előnye a gyors betanulás és a vizuális, átlátható felület, míg hátránya a korlátozott testreszabhatóság és a nagyobb projektekhez kevésbé megfelelő komplexitás. Az Acxor esetében a Trello inspirálta az intuitív, vizuális task lifecycle és board kezelést, amely egyszerre egyszerű és professzionális.

\SubSection{Asana}

Az Asana egy feladat- és projektmenedzsment platform, amely a csapatok együttműködését támogatja. Fő funkciói:

\begin{itemize}
\item Task és Subtask kezelés
\item Projektek és Timeline vizualizáció
\item Határidők és felelősök hozzárendelése
\item Dashboard és riporting
\item Integráció külső szolgáltatásokkal (Slack, Teams, GitHub, Google Workspace)
\end{itemize}

Előnye a könnyen átlátható felület és a gyors onboarding, hátránya a korlátozott workflow testreszabás és a részletes riporting hiánya. Az Acxor tervezésénél az Asana inspirációt jelentett a fókuszált task lifecycle és minimalista felület kialakításában.

\SubSection{Redmine}

A Redmine nyílt forráskódú rendszer, amely elsősorban testreszabhatóságáról és bővíthetőségéről ismert. Főbb jellemzői:

\begin{itemize}
\item Issue tracking és projekt hierarchia (főprojekt → alprojekt)
\item Testreszabható workflow és permission rendszer
\item Gantt és Calendar vizualizáció
\item Plugin és API alapú integráció
\end{itemize}

Előnye a nagy projektek kezelésére való alkalmasság és a nyílt forráskódú bővíthetőség, míg hátránya a kevésbé modern UI és a tanulási görbe. Az Acxorban a Redmine inspirációt jelentett a backend struktúrában és az issue hierarchia egyszerűsítésében, modern, letisztult felülettel.

\SubSection{Projektmenedzsment rendszerek összehasonlítása}

Az Acxor a közepes komplexitású fejlesztői csapatok számára készült, kiemelten figyelve a vizualitásra, egyszerűségre és motivációs rendszerekre. Összehasonlításképpen bemutatom a Redmine, Asana, Trello és Jira főbb jellemzőit:

\begin{table}[h!]
\centering
\scriptsize
\begin{tabular}{|l|c|c|c|}
\hline
\textbf{Funkció / Eszköz} & \textbf{Acxor} & \textbf{Trello} & \textbf{Jira} \\
\hline
Webes felület & Igen, modern vizuális UI & Igen, webes, vizuális & Igen, webes, komplex \\
Kanban board & Igen, beépített, drag\&drop & Igen, beépített & Részben, plugin / board szükséges \\
Sprint / Iteration & Részben, egyszerű & Nem & Igen, teljes körű \\
Gamifikáció & Igen, XP, szint, badge & Nem & Nem \\
Task / Subtask & Igen, egyszerű hierarchia & Igen, kártya / checklist & Igen, Epic→Story→Task→Subtask \\
Chat / Log & Igen, globális bot & Részben, kommentek & Részben, jegy alapú \\
Workflow testreszabás & Részben, egyszerű & Részben, limitált & Igen, részletes, szabály alapú \\
Permission rendszer & Részben, egyszerű & Részben, egyszerű & Igen, mély, szerepkör alapú \\
Integrációk & Igen, alap, kulcs szolgáltatások & Részben, Power-Ups & Igen, széles (Confluence, GitHub, Slack...) \\
\hline
\end{tabular}
\caption{Acxor összehasonlítása a Trello-val és a Jira-val}
\end{table}

\begin{table}[h!]
\centering
\scriptsize
\begin{tabular}{|l|c|c|c|}
\hline
\textbf{Funkció / Eszköz} & \textbf{Acxor} & \textbf{Redmine} & \textbf{Asana} \\
\hline
Webes felület & Igen, modern vizuális UI & Igen, webes, részletes & Igen, webes, intuitív \\
Kanban board & Igen, beépített & Részben, plugin szükséges & Igen, board / list nézet \\
Sprint / Iteration & Részben, egyszerű & Igen, teljes körű & Részben, idővonal / Timeline \\
Gamifikáció & Igen, XP, szint, badge & Nem & Nem \\
Task / Subtask & Igen, egyszerű hierarchia & Igen, issue / subtask & Igen, task / subtask \\
Chat / Log & Igen, globális bot & Részben, jegy alapú & Részben, kommentek \\
Workflow testreszabás & Részben, egyszerű & Igen, részletes & Részben, egyszerű \\
Permission rendszer & Részben, egyszerű & Igen, roles & Részben, alap \\
Integrációk & Igen, kulcs szolgáltatások & Részben, pluginok, SCM & Igen, Slack, GitHub, Google Workspace \\
\hline
\end{tabular}
\caption{Acxor összehasonlítása a Redmine-nel és az Asana-val}
\end{table}



\Section{Összegzés}

A fenti rendszerek áttekintése rávilágít arra, hogy a projektmenedzsment eszközök széles skálán helyezkednek el komplexitás, testreszabhatóság és felhasználói élmény szempontjából. Az Acxor célja a különböző rendszerek előnyeinek ötvözése: a JIRA funkcionalitásának, a Trello vizuális egyszerűségének, az Asana átlátható task lifecycle-jának, valamint a Redmine struktúrájának kombinálása egy intuitív, könnyen használható, ugyanakkor professzionális projektmenedzsment platformban.

Ez a koncepciói alap biztosítja, hogy a dolgozat későbbi fejezeteiben bemutatott fejlesztések jól meghatározott, átgondolt irányelv szerint történjenek, és az olvasó számára érthető, követhető legyen a választott megközelítés és technológiai háttér.

